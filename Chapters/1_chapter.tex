\chapterimage{head2.png} % Chapter heading image
\chapter{\normalsize Lecture 1: Transformers, Attention and Encoders}



\section{Sequence-to-Sequence Models \& Limitations of RNNs}

\subsection*{RNNs limitations (critical to understand):}
\begin{itemize}
    \item Cannot be trained in parallel (sequential processing)
    \item Past information influence is quickly lost over time
    \item Difficult to train due to vanishing/exploding gradients
    \item Formula chain for backpropagation:
    \[
    \frac{\partial h_t}{\partial h_0} = \left(\frac{\partial h_t}{\partial h_{t-1}}\right) \left(\frac{\partial h_{t-1}}{\partial h_{t-2}}\right) \cdots \left(\frac{\partial h_1}{\partial h_0}\right)
    \]
\end{itemize}


\section{Transformer Architecture}

\subsection*{Core innovation:}
Relies entirely on attention mechanisms without recurrence or convolutions

\subsection*{Structure:}
Encoder-Decoder architecture with multiple identical layers

\subsection*{Key components:}
\begin{itemize}
    \item Self-attention mechanism
    \item Multi-head attention
    \item Positional encoding
    \item Feed-forward networks
    \item Add \& Norm (residual connections + layer normalization)
\end{itemize}

\section{Self-Attention Mechanism}

\subsection*{Key equation:}
\[
\text{Attention}(Q, K, V) = \text{softmax}\left(\frac{QK^T}{\sqrt{d_k}}\right)V
\]

\subsection*{Components:}
\begin{itemize}
    \item \textbf{Q (Query)}: What we're looking for
    \item \textbf{K (Key)}: What we match against
    \item \textbf{V (Value)}: What we retrieve
\end{itemize}

\subsection*{Properties:}
Permutation-equivariant (needs positional encoding for order)

\begin{example}
\begin{quote}
\textit{"The animal didn't cross the street because it was too tired,"} self-attention helps \textbf{"it"} attend to \textbf{"animal"}
\end{quote}
\end{example}

\section{Multi-Head Attention}

\subsection*{Purpose:}
Allows the model to jointly attend to information from different representation subspaces

\subsection*{Process:}
\begin{enumerate}
    \item Project input into multiple sets of Q, K, V
    \item Perform attention on each set independently
    \item Concatenate results and project back to original dimension
\end{enumerate}

\subsection*{Formula:}
\[
\text{MultiHead}(Q, K, V) = \text{Concat}(\text{head}_1, \ldots, \text{head}_h) W^O
\]
\[
\text{where head}_i = \text{Attention}(Q W_i^Q, K W_i^K, V W_i^V)
\]

\section{Positional Encoding}

\subsection*{Purpose:}
Add sequence order information since transformer processes tokens in parallel

\subsection*{Original implementation:}
\begin{itemize}
    \item $\text{PE(pos, 2i)} = \sin\left(\frac{\text{pos}}{10000^{\frac{2i}{d_{\text{model}}}}}\right)$
    \item $\text{PE(pos, 2i+1)} = \cos\left(\frac{\text{pos}}{10000^{\frac{2i}{d_{\text{model}}}}}\right)$
\end{itemize}

\subsection*{Alternative:}
Learnable positional encodings (\texttt{nn.Parameter})

\subsection*{Application:}
Added to input embeddings before feeding to the model

\section{Transformer for Classification}

\subsection*{Architecture modifications:}
\begin{enumerate}
    \item Remove Transformer Decoder
    \item Add pooling at the Encoder output (avgpool or [CLS] token)
    \item Add a simple Linear Layer as ``decoder''
\end{enumerate}

\subsection*{[CLS] token approach:}
Special token that aggregates sequence information through self-attention


\section{BERT Overview}

\begin{itemize}
    \item \textbf{Full name:} Bidirectional Encoder Representations from Transformers
    \item \textbf{Pre-training tasks:}
    \begin{itemize}
        \item Masked Language Modeling (MLM)
        \item Next Sentence Prediction (NSP)
    \end{itemize}
    \item \textbf{Fine-tuning:} Add task-specific layers for classification, NER, etc.
\end{itemize}

\section*{Visual Explanations \& Examples to Remember}

\begin{enumerate}
    \item \textbf{YouTube Search as Q-K-V analogy:}
    \begin{itemize}
        \item Query: What you type in search bar
        \item Keys: Video metadata in database
        \item Values: The videos themselves
    \end{itemize}
    
    \item \textbf{Self-attention visualization} showing how words relate to each other:
    \begin{quote}
        Example: "The animal didn't cross the street because it was too tired" where \textbf{"it"} strongly attends to \textbf{"animal"}
    \end{quote}
    
    \item \textbf{Positional encoding heatmap:} Shows sine/cosine patterns that encode position information
\end{enumerate}

\section*{Key Connections to Emphasize}

\begin{enumerate}
    \item \textbf{RNNs $\rightarrow$ Transformers:} Transformers solve RNN limitations by replacing sequential processing with parallel attention
    \item \textbf{Self-attention $\rightarrow$ Multi-head attention:} Multi-head allows model to focus on different aspects of input simultaneously
    \item \textbf{Transformer $\rightarrow$ BERT:} BERT is essentially a pre-trained transformer encoder with specific training objectives
    \item \textbf{Attention $\rightarrow$ Classification:} How self-attention allows aggregation of information for classification tasks
\end{enumerate}

\section*{Quiz Questions for Self-Testing}

\begin{enumerate}
    \item \textbf{Q:} What is the primary advantage of transformer architecture over RNNs?\\
    \textbf{A:} Parallel processing of input sequences, allowing for more efficient training and better capture of long-range dependencies.

    \item \textbf{Q:} Why do we divide by $\sqrt{d_k}$ in the attention formula?\\
    \textbf{A:} To prevent the dot products from growing too large in magnitude, which would lead to extremely small gradients during backpropagation.

    \item \textbf{Q:} Explain the difference between "permutation-equivariant" and "permutation-invariant".\\
    \textbf{A:} Permutation-equivariant means the output changes in a corresponding way when the input order changes. Permutation-invariant means the output remains the same regardless of input order. Self-attention is permutation-equivariant.

    \item \textbf{Q:} Why is positional encoding necessary in transformers?\\
    \textbf{A:} Since transformers process all tokens in parallel with self-attention (which is permutation-equivariant), they need positional encoding to capture sequence order information.

    \item \textbf{Q:} How does the [CLS] token approach differ from average pooling for classification?\\
    \textbf{A:} The [CLS] token learns to aggregate sequence information through self-attention, while average pooling simply takes the mean of all token representations.

    \item \textbf{Q:} What are the components of a typical transformer encoder layer?\\
    \textbf{A:} Multi-head self-attention, Add \& Norm (residual connection and layer normalization), Feed-forward network, and another Add \& Norm.
\end{enumerate}

\vspace{1em}

\section*{Often Overlooked Details Examiners May Ask About}

\begin{enumerate}
    \item \textbf{Scaling factor in attention:} The $\sqrt{d_k}$ term is crucial for gradient stability and often overlooked.

    \item \textbf{Feed-forward networks in transformers:} These are applied position-wise and typically expand dimension then contract:
    \[
    \text{FFN}(x) = \max(0, xW_1 + b_1)W_2 + b_2
    \]

    \item \textbf{Masked self-attention in decoder:} Ensures predictions only depend on known outputs during training.

    \item \textbf{Implementation of multi-head attention:} Involves splitting dimensions rather than separate passes through the attention mechanism.

    \item \textbf{Residual connections:} Critical for training deep transformer networks by helping gradient flow.

    \item \textbf{Layer normalization vs. batch normalization:} Transformers use layer norm which normalizes across features rather than batch examples.

    \item \textbf{Computational complexity of self-attention:} $\mathcal{O}(n^2d)$ where $n$ is sequence length and $d$ is embedding dimension – quadratic in sequence length.
\end{enumerate}

\textbf{Note:} Examiners often ask you to derive or explain the attention mechanism in detail, so be prepared to write out the formula and explain each component's purpose!

